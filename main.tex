%%%%%%%%%%%%%%%%%%%%%%%%%%%%%%%%%%%%%%%%%%%%%%%%%%%%%%%%%%%%%%%%%%%%%%%%%%%%%%%%%%%
%% This project aims to create the UFC template for presentation.                %%
%% author: Maurício Moreira Neto - Doctoral student in Computer Science (MDCC)   %%
%% contacts:                                                                     %%
%%    e-mail: maumneto@ufc.br                                                    %%
%%    linktree: https://linktr.ee/maumneto                                       %%
%%%%%%%%%%%%%%%%%%%%%%%%%%%%%%%%%%%%%%%%%%%%%%%%%%%%%%%%%%%%%%%%%%%%%%%%%%%%%%%%%%%
\documentclass[compress, 10pt]{ctexbeamer}
% Inserting the preamble file with the packages
%%%%%%%%%%%%%%%%%%%%%%%%%%%%%%%%%%%%%%%%%%%%%%%%%%%%%%%%%%%%%%%%%%%%%
%% This file contains the packages that can be used in the beamer. %%
%%%%%%%%%%%%%%%%%%%%%%%%%%%%%%%%%%%%%%%%%%%%%%%%%%%%%%%%%%%%%%%%%%%%%
% Package to fonts family
\usepackage[T1]{fontenc}
% Package to accentuation
\usepackage[utf8]{inputenc}
% Package to Portuguese language
\usepackage[english]{babel}
% Package to Figures
\usepackage{graphicx}
% Package to the colors
\usepackage{color}
% Package to the colors
\usepackage{xcolor}
% Packages to math symbols and expressions
\usepackage{amsfonts, amssymb, amsmath}
% Package to multiple lines and columns in table
\usepackage{multirow, array} 
% Package to create pseudo-code
% For more detail of this package: http://linorg.usp.br/CTAN/macros/latex/contrib/algorithm2e/doc/algorithm2e.pdf
\usepackage{algorithm2e}
% Package to insert code
\usepackage{listings} 
\usepackage{keyval}
% Package to justify text
\usepackage[document]{ragged2e}
% Package to manage the bibliography
\usepackage[backend=biber, style=numeric, sorting=none]{biblatex}
% Package to facilities quotations
\usepackage{csquotes}
% Package to use multicols
\usepackage{multicol}
\usepackage{minted}

\usepackage{hyperref}
\newcommand{\email}[1]{
    \texttt{
      \href{mailto:#1}{#1}
    }
}

% Beamer settings
% \metroset{progressbar=none}
\setbeamerfont{title}{size=\huge, series=\bfseries}
\setbeamerfont*{subtitle}{size=\large, shape=\itshape}
\setbeamerfont{section title}{size=\Large, series=\bfseries}
\setbeamerfont{frametitle}{size=\large, series=\bfseries}
\setbeamerfont{caption}{size=\footnotesize, series=\bfseries}
\setbeamerfont{footnote}{size=\tiny}
\setbeamerfont{alerted text}{series=\bfseries}
\addtobeamertemplate{institute}{\raggedleft}{}
\setbeamertemplate{title}{%
  \raggedleft
  \linespread{1.0}%
  \inserttitle
  \hspace*{1.2cm}\par
  \vspace*{0.5em}}
\setbeamertemplate{subtitle}{%
  \raggedleft
  \insertsubtitle
  \hspace*{1.2cm}\par
  \vspace*{0.5em}}
\setbeamertemplate{title page}{%
  \begin{minipage}[b]{\textwidth}
    \usebeamertemplate*{title graphic}\vfill
    \usebeamertemplate*{title}
    \usebeamertemplate*{subtitle}
    \usebeamertemplate*{title separator}
    \usebeamertemplate*{author}
    \usebeamertemplate*{date}
    \usebeamertemplate*{institute}
    \vfill
  \end{minipage}}
\setbeamertemplate{frame numbering}{\zhnumber[style=Financial]{\insertframenumber}}
\setbeamertemplate{itemize/enumerate subbody begin}{\footnotesize}
\setbeamertemplate{caption}{\parbox{\textwidth}{\centering\insertcaption}\par}
% Inserting the references file
\addbibresource{references.bib}

\usetheme{Xiaoshan}
\usefonttheme{serif,professionalfonts}

% Title
\title[如何使用 \LaTeX]{\huge \textbf{ 如何使用 \LaTeX}}
% Subtitle
\subtitle{一些例子}
% Author of the presentation
\author{李子强 \enskip 计算机系}
\institute{\raggedleft 南方科技大学图书馆}
\date{2021年3月}

\logo{\includegraphics[scale=0.2]{libs/LOGO3_small.png}\hspace{9.3cm} \vspace{-1cm}}


%%%%%%%%%%%%%%%%%%%%%%%%%%%%%%%%%%%%%%%%%%%%%%%%%%%%%%%%%%%%%%%%%%%%%%%%%%%%%%%%%%
%% Start Document of the Presentation                                           %%               
%%%%%%%%%%%%%%%%%%%%%%%%%%%%%%%%%%%%%%%%%%%%%%%%%%%%%%%%%%%%%%%%%%%%%%%%%%%%%%%%%%
\begin{document}
% insert the code style
\input{libs/code_style}

%% ---------------------------------------------------------------------------
% First frame (with tile, subtitle, ...)

{%
    \usebackgroundtemplate{\includegraphics[width=1.7\paperwidth]{libs/LOGO transparent.png}}
    \begin{frame}
    \maketitle
    \end{frame}%
}

%% ---------------------------------------------------------------------------
% Second frame

\begin{frame}{中文测试}
    中文测试

    \emph{强调文本}

    \alert{内容永远比格式重要!}

\end{frame}


% %% ---------------------------------------------------------------------------
% % This presentation is separated by sections and subsections
\section{第一部分}

\begin{frame}{《新青年》发刊词}
    % itemize
    \emph{敬告青年}
    \begin{enumerate}
        \item 自主的而非奴隶的
        \item 进步的而非保守的
        \item 进取的而非退隐的
        \item 世界的而非锁国的
        \item 实利的而非虚文的
        \item 科学的而非想象的
    \end{enumerate}

\end{frame}

% %% ---------------------------------------------------------------------------
\subsection{分块}
\begin{frame}{分块}
    % Blocks styles
    \begin{block}{分块 I}
        Text in the 分块.
    \end{block}

    \begin{alertblock}{分块 II}
        Text in the alert 分块.
    \end{alertblock}

    \begin{exampleblock}{分块 III}
        Text in the example 分块.
    \end{exampleblock}
\end{frame}


% %% ---------------------------------------------------------------------------
\subsection{Express programs}
\begin{frame}{Insert Algorithm}
    \begin{algorithm}[H]
        \SetAlgoLined
        \LinesNumbered
        \SetKwInOut{Input}{input}
        \SetKwInOut{Output}{output}
        \Input{x: float, y: float}
        \Output{r: float}
        \While{True}{
          r = x + y\;
          \eIf{r >= 30}{
           ``O valor de $r$ é maior ou iqual a 10.''\;
           break\;
           }{
           ``O valor de $r$ = '', r\;
          }
         } 
         \caption{Algorithm Example}
    \end{algorithm}
\end{frame}

% %% ---------------------------------------------------------------------------

\begin{frame}{Insert code}
    \inputminted[
        linenos=true,
        breaklines, 
        breakafter=d,
        bgcolor=codebackground
    ]{Python}{code/main.py}
\end{frame}

%% ---------------------------------------------------------------------------
\begin{frame}{Insert code}
    \inputminted[
        linenos=true,
        breaklines, 
        breakafter=d,
        bgcolor=codebackground
    ]{C}{code/source.c}
\end{frame}

%% ---------------------------------------------------------------------------
\begin{frame}{Insert cod}
    \lstinputlisting[language=Java]{code/helloworld.java}
\end{frame}

%% ---------------------------------------------------------------------------
\begin{frame}{Insert cod}
    \lstinputlisting[language=HTML]{code/index.html}
\end{frame}

%% ---------------------------------------------------------------------------
% This frame show an example to insert multicolumns

%% ---------------------------------------------------------------------------
% This frame show an example to insert figures
\section{Images}
\begin{frame}{Section III - 图片}
    \begin{figure}
        \centering
        \caption{SUSTech LOGO.}
        \includegraphics[scale=0.3]{libs/LOGO.png}
        \label{fig:ufc_emblem}
    \end{figure}
\end{frame}

%% ---------------------------------------------------------------------------
\section{测试引用}
\begin{frame}{引用}
    引用爱因斯坦\cite{einstein}。
    
    “真理只有一个,而究竟谁发现了真理,不依靠主观的夸张,而依靠客观的实践。” -- 毛泽东\cite{毛泽东1949新民主主义论}。
\end{frame}

%% ---------------------------------------------------------------------------
% Reference frames
\begin{frame}[allowframebreaks]{References}
    % \frametitle{References}
    \printbibliography
\end{frame}

% %% ---------------------------------------------------------------------------
% % Final frame

{
    \usebackgroundtemplate{\includegraphics[width=1.7\paperwidth]{libs/LOGO transparent.png}}
    \begin{frame}
        \centering
        {\Huge \bfseries \emph{Thanks}}
        \vspace{1cm}

        {\Large \bfseries Contact:}

        \vspace{0.5cm}
        {\large \bfseries \email{Cnatact info}}
    \end{frame}
}

\end{document}